\subsection{Choice of Solver}

The Automatic Placement of Rulesets problem involves intricate dependencies and memory constraints. For this reason, choosing the right solver was critical to the project's success. We chose to formulate the problem as a MaxSAT instance because these solvers excel in managing optimization problems, allowing for a formulation of the objective functions through weighted clauses.

MaxSAT is an optimization version of the Boolean Satisfiability problem (SAT). In contrast to SAT, MaxSAT seeks an assignment that satisfies the maximum number of clauses. In MaxSAT, clauses are divided into two categories:

\begin{enumerate}
    \item \textbf{Hard Clauses:} Mandatory constraints that must be satisfied for a solution to be valid. They represent non-negotiable requirements and ensure the fundamental feasibility of the solution. In our problem, all the constraints, except for the minimization of re-circulations, were considered hard clauses.

    \item \textbf{Soft Clauses:} Optional constraints that contribute to the optimization objective. They represent preferences or desires in the problem. Satisfying soft clauses enhances the solution's quality but is not obligatory for validity. The solver's job is to find an assignment that optimizes the total weight of satisfied soft clauses which, in our case, was the minimization of re-circulations.
\end{enumerate}

To conclude, MaxSAT's ability to handle both mandatory and optimality constraints, makes it the superior choice for complex problems like the Automatic Placement of Rulesets. After some experimentation, we decided to use RC2 solver.

\subsection{Encoding of the Problem}

\subsubsection{Variables}

The first step in the formulation of the problem was to identify the variables that would be used to represent the problem's components. The placement of a switch \(i\) in position \(j\) was represented by the variable \(sw[i][j]\). Similarly, the placement of a group \(k\) in stage \(j\) of switch \(i\) was represented by the variable \(gr[i][j][k]\).

\subsubsection{Auxiliary Variables}

In addition to the variables representing the placement of switches and groups, we introduced auxiliary variables to facilitate the formulation of constraints. The first set of auxiliary variables \(besw[i][j]\) was used to indicate whether switch \(i\) is behind switch \(j\). The second set of auxiliary variables \(begr[i][j]\) was used to indicate whether group \(j\) is behind group \(i\). Importantly, in our formulation, for a group to be behind another group, it must be in a switch that is behind the switch of the other group or, if the groups are in the same switch, it must be in the same stage or in a stage that comes before the stage of the other group.

To make the values of these variables meaningful and consistent with the main variables, we formulted constraints that ensured such properties:

\begin{enumerate}
    \item \textbf{Relationship between \texttt{sw} and \texttt{besw}:} For each pair of switches with their respective positions \((s1, p1)\) and \((s2, p2)\), such that \(p1 < p2\), the following clause was added:
    \begin{center}
        \((sw[s1][p1] \cap sw[s2][p2]) \Rightarrow besw[s1][s2]\)
    \end{center}

    \item \textbf{Relationship between \texttt{gr} and \texttt{begr}:} For each pair of groups with their respective switches and stages \((s1, st1, g1)\) and \((s2, st2, g2)\), such that \(s1 = s2\) and \(st1 \leq st2\), the following clause was added:
    \begin{center}
        \((gr[s1][st1][g1] \cap gr[s2][st2][g2]) \Rightarrow begr[g1][g2]\)
    \end{center}
    Importantly, we did not need to add clauses for the case where \(s1 \neq s2\) because it is already ensured by the relationship between \texttt{sw} and \texttt{besw}.
\end{enumerate}

\subsubsection{Hard Constraints}

Below, we delineate the non-negotiable constraints that must be satisfied for a solution to be valid. These constraints were formulated as hard clauses, ensuring the fundamental feasibility of the solution.

\begin{enumerate}
    \item \textbf{Uniqueness Constraints:} Constraints were formulated to ensure that each switch and group of rules was placed exactly once. Utilizing cardinality encoding, we imposed uniqueness properties on the variables \(sw\) and \(gr\). This was accomplished through the use of the \texttt{CardEnc.equals} method with a cardinality of 1. Moreover, an additional set of cardinality constraints is needed to ensure that only one switch is placed in each position. This was achieved through the use of the \texttt{CardEnc.atmost} method with a cardinality of 1.

    \item \textbf{Memory Limit Compliance:} Constraints were devised to ensure that the total memory requirements of all groups within each stage did not exceed the stage's memory capacity. Employing Pseudo-Boolean Encodings through the use of \texttt{PBEnc.atmost} method, we formulated constraints that ensure that the sum of memory requirements of the groups in each stage does not surpass that stage's memory limit.

    \item \textbf{Dependency Handling:} Dependencies between groups need to be handled in order to prevent re-circulations between switches. For each pair of groups \((g1, g2)\) where \(g2\) depends on \(g1\), specific conditions were established. Using the auxiliary variables \(besw\) and \(gr\), this is fairly straightforward. For each pair of groups \((g1, g2)\) where \(g2\) depends on \(g1\) and for each pair of switches \((s1, s2)\) where \(s2 > s1\) and for each pair of stages \((st1, st2)\), the following clauses were added:
    \begin{center}
        \((besw[s1][s2] \cap gr[s2][st2][g1])\Rightarrow \lnot gr[s1][st1][g2]\)

        \((besw[s2][s1] \cap gr[s1][st1][g1])\Rightarrow \lnot gr[s2][st2][g2]\)
    \end{center}
\end{enumerate}

\subsubsection{Soft Constraints}    

The only soft constraint in the problem was the minimization of re-circulations. Through the use of the auxiliary variables \(begr\), we formulated the following clause for each pair of groups \((g1, g2)\) where \(g2\) depends on \(g1\):
\begin{center}
    \(\lnot begr[g1][g2]\)
\end{center}

Through this encoding of constraints, we transformed the problem into a series of boolean variables and logical relationships. The resulting formula encapsulated the essence of the problem, enabling sophisticated solvers like RC2 to explore the solution space efficiently.