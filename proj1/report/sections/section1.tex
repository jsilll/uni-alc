In the rapidly evolving realm of cloud computing, the intricate orchestration of network traffic has become a linchpin for seamless operations. Large-scale cloud providers heavily rely on virtual switches to make critical decisions regarding the handling of network packets. These switches employ a series of predefined rules, ranging from exact to pattern-based matching, to determine the fate of each packet – whether it should be forwarded, modified, or dropped. The challenge lies in not just formulating these rules comprehensively but also in their optimal arrangement to maximize network performance.

This project embarks on a mission to tackle this challenge head-on, with the development of an innovative software tool designed to solve the Automatic Placement of Rulesets problem. At its core, this initiative aims to streamline the process of associating rules with packets, ensuring their efficient traversal through the network infrastructure. To achieve this, the project harnesses the computational prowess of advanced solvers, specifically tailored for Satisfiability (SAT), Maximum Satisfiability (MaxSAT), Pseudo-Boolean Satisfiability (PBS), and Pseudo-Boolean Optimization (PBO). By integrating these cutting-edge tools, our objective is to revolutionize how virtual switches handle packets, augmenting both precision and speed in decision-making.

In the subsequent sections, this report delves deeply into the nuances of the Automatic Placement of Rulesets problem. It offers an in-depth exploration of the contextual landscape, elucidating the challenges inherent in this endeavor. The report further unravels the methodologies and algorithms deployed, shedding light on the intricate decision-making processes and the pivotal role played by the aforementioned solvers. Moreover, it discusses the intricacies of implementation, the hurdles encountered, and the outcomes attained, providing a comprehensive panorama of the project's scope and significance.

This project not only seeks to enhance the operational efficiency of virtual switches but also aims to make a significant contribution to the broader discourse on network optimization within cloud computing. By pioneering inventive solutions to the Automatic Placement of Rulesets problem, this initiative anticipates paving the way for more agile, responsive, and efficient cloud networks. In doing so, it heralds a transformative era in digital connectivity and data management.

In the context of this project, specific assumptions lay the groundwork for our exploration of the Automatic Placement of Rulesets problem. Firstly, groups of rules are designated with unique identifiers, commencing with the number 1 for the initial group. Additionally, the rules within a group are structured in a manner that directs the next step in the packet's journey through the network. This sequencing ensures that there are no cyclical dependencies in the ruleset, preventing any ambiguous or infinite processing loops.

The practical implementation of these rulesets occurs within the framework of programmable switches, utilizing the Protocol Independent Switch Architecture (PISA) as illustrated in Figure 2. Within this architecture, incoming packets traverse through the Parser, responsible for dissecting the packet's header fields. Subsequently, the packet progresses through the match+action pipeline, which consists of distinct stages. Each stage serves as a discrete hardware region within the switch, both logically and physically separated. Importantly, these stages are not homogenous in terms of rule processing capacity; instead, they impose limitations on the number of rules they can accommodate based on available resources.

This partitioning of the pipeline into stages, each capable of hosting multiple groups of rules, introduces a crucial dimension to our problem. Efficient placement of rules within these stages is pivotal for ensuring swift and accurate packet processing. It is this intricate interplay between rule groups, stages, and the limited resources therein that underscores the complexity of the Automatic Placement of Rulesets problem. As we delve deeper into this project, we will dissect these complexities, leveraging advanced solvers and computational techniques to arrive at optimal solutions. By navigating these intricacies, our endeavor is to not only enhance the efficiency of virtual switches but also contribute valuable insights to the broader domain of network architecture and protocol-independent switching.

When a packet enters the pipeline, it follows a linear progression from one stage to the next, adhering to a strict forward movement. Each stage permits only one table lookup, a restriction that ensures robust performance guarantees but limits the system's programmability. While a packet cannot revert to a previous stage, it can be redirected back to the pipeline's outset, a process known as re-circulation. However, this corrective measure comes at the expense of performance; the more re-circulations occur, the lower the network's efficiency. In the absence of re-circulations, the system operates at its peak performance, commonly referred to as line-rate.